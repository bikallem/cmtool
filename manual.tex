\documentclass[10pt]{article}
\usepackage{code,url}

\topmargin=0in
\headheight=0in
\headsep=0in
\textheight=9in
\oddsidemargin=0in
\evensidemargin=0in
\textwidth=6.5in

\parindent=0in
\parskip=10pt

\title{CM-Lex and CM-Yacc User's Manual \\ \Large Version 0}
\author{Karl Crary \\ Carnegie Mellon University}
\date{}

\newcommand{\nonterm}[1]{\mbox{$\langle${\it{}#1}$\rangle$}}
\newcommand{\lb}{$[$}
\newcommand{\rb}{$]$}
\newcommand{\OR}{{\it{}or}}


\begin{document}
\maketitle

\section{CM-Lex}

CM-Lex is a lexer generator for Standard ML and Haskell.  It converts
a specification file into an SML or Haskell source-code file that
implements a lexer functor.  The programmer may then instantiate that
functor with appropriate code for lexing actions to obtain a lexer.

\subsection{The specification file}

A CM-Lex specification consists of a series of directives:

\begin{itemize}


\item
\cd{name \nonterm{identifier}}

Specifies the name of the generated functor for SML lexers,
and the name of the generated module for Haskell lexers.  This is
a required directive.

In Haskell specifications, the name must be a valid module
identifier, so it must begin with a capital letter.  In SML
specifications the name is not required to be capitalized, but it
is standard practice in SML code to capitalize functor names
nonetheless.

\item
\cd{alphabet \nonterm{alphabet size}}

Specifies the size of the input alphabet, which must be a positive
integer.  The generated lexer will not recognize any symbols whose
ordinal is greater or equal to the given value.  This is a
required directive and it must appear before any {\tt set}, {\tt regexp},
or {\tt function} directives.

\item
\cd{set \nonterm{name} = \nonterm{charset}}

Declares a character set \nonterm{name} with the definition \nonterm{charset}.

\item
\cd{regexp \nonterm{name} = \nonterm{regexp}}

Declares a regular expression \nonterm{name} with the definition \nonterm{regexp}.

\item
\cd{function \nonterm{function name} : \nonterm{type name} = \nonterm{clauses}}

Specifies a lexing function \nonterm{function name} that has type
\nonterm{type name} and is defined by \nonterm{clauses}, where each
clause has the form:

\begin{code}
\nonterm{regexp} => \nonterm{action name}
\end{code}

\item
{\tt monadic}

{\em Haskell specifications only:}
Specifies that CM-Lex should generate a monadic lexer.
\end{itemize}

Comments are written ``\cd{/* {\it{}comment text} */}'' and may appear
anywhere in the specification.  Comments may be nested.

\subsubsection{Character sets and regular expressions}

For the purposes of CM-Lex, symbols are identified with nonnegative
integers.  Typically the number is the ASCII code of a character, but
establishing the correspondence is left to the programmer.

\begin{tabbing}
\quad \= \cd{(range \nonterm{range-spec} \dots)} \quad\= \kill
\nonterm{symbol} ::=
\\ \>
\nonterm{number}  \>
the indicated number
\\ \>
\cd{'} \nonterm{printable character} \>
the ASCII code of the indicated character
\end{tabbing}

Character sets and regular expressions are specified using the Scheme
Shell's SRE notation:

\begin{tabbing}
\quad \= \cd{(range \nonterm{range-spec} \dots)} \quad\= \kill
\nonterm{charset} ::=
\\[1ex] \>
\nonterm{identifier} \>
the indicated character set
\\[1ex] \>
\nonterm{symbol} \>
the singleton set containing \nonterm{symbol}
\\[1ex] \>
\cd{empty} \>
the empty set
\\[1ex] \>
\cd{any} \>
the universal set (all symbols up to $N-1$, where $N$ is the alphabet size)
\\[1ex] \>
\cd{(or \nonterm{charset} \dots)} \>
set union
\\ \>
\cd{(| \nonterm{charset} \dots)}
\\[1ex] \>
\cd{(and \nonterm{charset} \dots)} \>
set intersection
\\ \>
\cd{(& \nonterm{charset} \dots)}
\\[1ex] \>
\cd{(- \nonterm{charset} \dots)} \>
set difference: the first set minus all the following sets
\\[1ex] \>
\cd{(\char"7E{} \nonterm{charset} \dots)} \>
set complement: equivalent to (- any (or \nonterm{charset} \dots))
\\[1ex] \>
\cd{(range \nonterm{range-spec} \dots)} \>
union of the specified ranges
\\ \>
\cd{(/ \nonterm{range-spec} \dots)}
\end{tabbing}

\begin{tabbing}
\quad \= \cd{(range \nonterm{range-spec} \dots)} \quad\= \kill
\nonterm{range-spec} ::=
\\ \>
\nonterm{symbol1} \nonterm{symbol2} \>
the range of symbols from \nonterm{symbol1} to \nonterm{symbol2} (inclusive)
\end{tabbing}

\begin{tabbing}
\quad \= \cd{(** \nonterm{number1} \nonterm{number2} \nonterm{regexp})} \quad\= \kill
\nonterm{regexp} ::=
\\[1ex] \>
\nonterm{identifier} \>
the indicated character set or regular expression
\\[1ex] \>
\nonterm{symbol} \>
match the indicated symbol
\\[1ex] \>
\cd{any} \>
match any character
\\[1ex] \>
\cd{epsilon} \>
match the empty string
\\[1ex] \>
\cd{empty} \>
match nothing
\\[1ex] \>
\cd{eos} \>
match end-of-stream
\\[1ex] \>
\cd{" \nonterm{string} "} \>

\parbox[t]{4in}{match the indicated string: that is, the sequence of symbols given by
the ASCII codes of the characters of \nonterm{string} (the string can \\
contain any printable character other than a quotation mark)}
\\[1ex] \>
\cd{(seq \nonterm{regexp} \dots)} \>
concatenation
\\ \>
\cd{(: \nonterm{regexp} \dots)}
\\[1ex] \>
\cd{(or \nonterm{regexp} \dots)} \>
choice
\\ \>
\cd{(| \nonterm{regexp} \dots)}
\\[1ex] \>
\cd{(? \nonterm{regexp})} \>
zero or one matches
\\[1ex] \>
\cd{(* \nonterm{regexp})} \>
zero or more matches
\\[1ex] \>
\cd{(+ \nonterm{regexp})} \>
one or more matches
\\[1ex] \>
\cd{(= \nonterm{number} \nonterm{regexp})} \>
exactly \nonterm{number} matches
\\[1ex] \>
\cd{(>= \nonterm{number} \nonterm{regexp})} \>
at least \nonterm{number} matches
\\[1ex] \>
\cd{(** \nonterm{number1} \nonterm{number2} \nonterm{regexp})} \>
between \nonterm{number1} and \nonterm{number2} matches (inclusive)
\end{tabbing}


\subsubsection{Function specifications}

A lexing function is specified by the directive:

\begin{code}
function \nonterm{function name} : \nonterm{type name} =
   \nonterm{regexp} => \nonterm{action}
   \dots
\end{code}

This generates a function named \nonterm{function name} with return type
\nonterm{type name}.  That function finds the longest prefix of its input
stream that matches any of the given regular expressions and activates
the corresponding action.  If the longest possible match matches two
different regular expressions, CM-Lex prefers the one that appears
earlier.

If a function is inexhaustive ({\em i.e.,} if it is possible for an input
stream to satisfy none of the regular expressions), CM-Lex adds a
default action that raises an exception.  The programmer can add an
explicit default action by adding a clause \cd{epsilon => my_default}
to the end of the clause list.  This ensures that the function is
exhaustive, because it always can at least match the empty prefix.

An action name can be used more than once, either in the same function
specification, or even in different function specifications, provided
that the functions have the same return type.


\subsection{Invocation}

CM-Lex is invoked from the command line with the specification file's
name as its argument.  The desired output file name can be specified
with the \cd{-o} flag.  Otherwise, the output file name is derived from from
the input file name by appending the appropriate suffix.

\begin{code}
cmlex foo.cmlex               # generates foo.cmlex.sml
cmlex foo.cmlex -o bar.sml    # generates bar.sml

cmlex-hs foo.cmlex            # generates foo.cmlex.hs
cmlex-hs foo.cmlex -o bar.hs  # generates bar.hs
\end{code}


\subsection{Streams (Standard ML)}

The generated lexer takes its input in the form of a lazy stream,
which is a possibly-infinite list in which each element is computed
the first time it is needed, and then memoized for future uses.

A stream type is included in CM-Lib, a library packaged with CM-Lex
and CM-Yacc.  The module Stream implements the signature {\tt STREAM}, an
abbreviated version of which appears below:

\begin{code}
signature STREAM =
   sig
      type 'a stream
      datatype 'a front = Nil | Cons of 'a * 'a stream

      val front : 'a stream -> 'a front
      val lazy : (unit -> 'a front) -> 'a stream
   end
\end{code}


The {\tt front} function forces the computation of the next element of the
stream, if any.  It returns a front, which exposes whether the stream
is empty, and, if not, gives its next element and the remainder of the
stream.  The {\tt lazy} function builds a stream from a thunk returning
the stream's front.  A thunk is used so that the front need not be
computed until the stream is forced.

The {\tt STREAM} signature also includes a variety of other functions for
building and manipulating streams.  For example, {\tt fromString} and
{\tt fromTextInstream} build character streams that read input from a string
and from an SML IO stream, respectively:

\begin{code}
val fromString : string -> char stream
val fromTextInstream : TextIO.instream -> char stream
\end{code}

A lexer need not use streams, it may use any type that implements the
type class {\tt STREAMABLE}, which specifies types that are compatible with
the stream forcing operation, {\tt front}:

\begin{code}
signature STREAMABLE =
   sig
      type 'a t

      datatype 'a front = Nil | Cons of 'a * 'a t
      val front : 'a t -> 'a front
   end
\end{code}

The canonical implementation of {\tt STREAMABLE} is {\tt
StreamStreamable}, which implements {\tt t} as simply {\tt
Stream.stream}.  However, the programmer may also use {\tt
ListStreamable}, which implements {\tt t} as {\tt list}, or any other
way he or she chooses.


\subsection{The generated functor (Standard ML)}
\label{sec:lex-functor-sml}

CM-Lex generates a functor from the given specification file.  For example,
consider the specification:

\begin{code}
name LexerFun
alphabet 128

function foo : t =
  (* 'a) => astar

function bar : u =
  (* 'b) => bstar
\end{code}


CM-Lex generates the functor:


\begin{strictcode}
functor LexerFun
   (structure Streamable : STREAMABLE
    structure Arg :
       sig
          type symbol
          val ord : symbol -> int

          type t
          type u

          type self = { bar : symbol Streamable.t -> u,
                        foo : symbol Streamable.t -> t }
          type info = { match : symbol list,
                        len : int,
                        start : symbol Streamable.t,
                        follow : symbol Streamable.t,
                        self : self }

          val astar : info -> t
          val bstar : info -> u
       end)
   :>
   sig
      val bar : Arg.symbol Streamable.t -> Arg.u
      val foo : Arg.symbol Streamable.t -> Arg.t
   end
= itdots implementation dots
endstrictcode


The functor takes in two arguments.  The first structure argument is
an implementation of of the {\tt STREAMABLE} type class for the lexer to
use.  This is typically {\tt StreamStreamable} (so {\tt Streamable.t} =
{\tt Stream.stream}), but need not be.

The second structure argument, {\tt Arg}, implements the lexer's
actions.  First, the programmer defines the type of the symbols that
the lexer processes.  This is typically {\tt char}, but need not be.
Whatever symbol type is chosen, {\tt ord} gives the coercion from
symbols to the numbers that CM-Lex uses to identify them.

Next, the programmer implements the types used for the functions'
result types.  The specification uses two type names, {\tt t} and {\tt u}, so
the functor demands {\tt t} and {\tt u} as type arguments.

Next are two type definitions we will return to in a moment, and
finally the programmer implements the lexing actions.  The
specification uses two action names, {\tt astar} and {\tt bstar}, so the 
functor demands {\tt astar} and {\tt bstar} as value arguments.

Since {\tt astar} was used as an action for {\tt foo}, which was declared to
return type {\tt t}, the {\tt astar} action must return type {\tt t}.  Similarly,
{\tt bstar} must return type {\tt u}.

Once given the two structure arguments, the generated functor returns
a structure containing the two lexing functions, {\tt foo} and {\tt bar}.
Each one takes in an {\tt Arg.symbol Streamable.t} (typically a {\tt char
stream}), and returns the appropriate type.

\subsubsection{The match info}
\label{sec:match-info}

When a lexing action is invoked by the lexer, it is passed a record
containing various matching information:

\begin{itemize}
\item
{\tt match}:
the actual symbols that were matched.

\item
{\tt len}:
the length of match.

\item
{\tt follow}:
the remaining stream after the matched symbols.

\item
{\tt start}:
the stream on which the lexer was called ({\em i.e.,} beginning with the
matched symbols).  This is useful for re-lexing some or all of the
stream, as in Lex's {\tt yyless} directive.

\item
{\tt self}:
a record containing the lexing functions being defined, so that
actions can reinvoke the lexer.
\end{itemize}

Note that the lexing functor is {\em stateless}.  Nowhere does it maintain
the current input stream.  Rather, it is the job of the programmer,
through the lexing actions, to manage the input stream.

Typically, an action calls the lexer (through the {\tt self} argument) on
the remaining stream ({\em i.e.,} the {\tt follow} argument).  However, the
programmer may manipulate the stream as he or she desires.  For
example, one may push additional symbols onto the stream (as in Lex's
{\tt yymore} directive), simply by doing so before calling the lexer.


\subsection{Streams (Haskell)}

As in Standard ML, Haskell lexers use a type class {\tt Streamable} that
specifies types compatible with the stream forcing operation {\tt front}:

\begin{code}
data Front a b =
   Nil 
 | Cons a b

class Monad m => Streamable s m where
   front :: s a -> m (Front a (s a))
\end{code}


A type constructor {\tt s} is streamable (with an associated monad
{\tt m}) if it provides a {\tt front} operation that takes an {\tt s
a} and then --- in the monad {\tt m} --- returns a stream front with head {\tt
a} and tail {\tt s a}.  This differs from the {\tt STREAMABLE} class
in Standard ML, which makes no explicit mention of a monad since SML
permits side-effects.

One important instance is lists.  Lists are purely functional
streamables, and thus belong to {\tt Streamable} with any monad (but
particularly with {\tt Identity}).

For effectful streams, the canonical instance is the {\tt Stream} type:

\begin{code}
newtype Stream m a =
   Stream (m (Front a (Stream m a)))
\end{code}

For example, streams that result from reading input from a file will
typically have the type \cd{Stream Char IO}.  Forcing such a stream
will result in an IO effect and return a \cd{Front} containing a
character and the rest of the stream.  The effect occurs because it
might be necessary to read the file to obtain the next character.

A {\tt Stream} can be formed from a {\tt Front} in two ways:

\begin{code}
eager :: Monad m => Front a (Stream m a) -> Stream m a
lazy  :: MonadMemo m => m (Front a (Stream m a)) -> m (Stream m a)
\end{code}

When a front is held outright, \cd{eager} coerces it to a stream.
However, when one has a front under a monad (the typical case for
monadic lexers), one uses \cd{lazy}, which memoizes the monadic front
before coercing it to a stream.  The memoization means that if the
stream is forced more than once, the monadic operation is not
repeated; instead the prior result is recalled.  This requires that
the monad belong to the \cd{MonadMemo} class, which includes
\cd{Identity}, \cd{IO}, and \cd{ST}, as well as every monad belonging
to \cd{MonadIO}.

The \cd{Stream} module contains a variety of other functions for
building and manipulating streams.  For example, fromList and
fromHandle build character streams that read input from a list (or
string) and from an IO handle, respectively:

\begin{code}
fromList :: Monad m => [a] -> Stream m a
fromHandle :: Handle -> IO (Stream IO Char)
\end{code}



\subsection{The generated functor (Haskell)}
\label{sec:lex-functor-hs}

Instantiating the generated code for Haskell is conceptually similar
to the process for Standard ML.  However, Haskell, unlike Standard ML,
does not support functors.  Instead, the tool generates an ersatz
functor using polymorphic functions.  For example, again consider the
specification from Section~\ref{sec:lex-functor-sml}:

\begin{code}
name LexerFun
alphabet 128

function foo : t =
  (* 'a) => astar

function bar : u =
  (* 'b) => bstar
\end{code}


CM-Lex generates the functor:\footnote{For readability, we have
used unqualified names, but the tool actually uses fully qualified names.}


\begin{strictcode}
data Arg stream symbol t u =
   Arg { ord :: symbol -> Int,

         {- type arguments -}
         t :: t,
         u :: u,

         {- action arguments -}
         astar :: LexInfo stream symbol -> t,
         bstar :: LexInfo stream symbol -> u }

foo :: Streamable stream Identity
       => Arg stream symbol t u -> stream symbol -> t
bar :: Streamable stream Identity
       => Arg stream symbol t u -> stream symbol -> u
endstrictcode


To use the functor, one constructs an inhabitant of the {\tt Arg}
type.  As in the Standard ML version, an argument contains an {\tt
ord} function that converts symbols to integers, implementations of
the type arguments ({\tt t} and {\tt u}) in this example, and
implementations of the actions ({\tt astar} and {\tt bstar} in this
example).  One then calls the functions ({\tt foo} and {\tt
bar}) with the argument to obtain the lexing functions.

The type arguments are given by dummy fields.  The terms put into the
dummy fields are not used; their purpose is solely to specify the
types.  Thus, they can be filled in with \cd{undefined :: T} for
some {\tt T}.

Each action is passed a {\tt LexInfo}, which is defined as follows:

\begin{strictcode}
data LexInfo stream symbol =
   LexInfo
   { match :: [symbol],
     len :: Int,
     start :: stream symbol,
     follow :: stream symbol }
endstrictcode

The information in the {\tt LexInfo} is similar to that from
Section~\ref{sec:match-info}.  However, note that {\tt LexInfo} does
not include a {\tt self} field.  In Haskell a {\tt self} field is not
necessary: since anything can call anything else, no special mechanism
is necessary for an action to reinvoke the lexer.


\subsubsection{Monadic lexers}

If the {\tt monadic} directive is added to the specification, CM-Lex
generates a monadic lexer.  In a monadic lexer, functor takes the
following form:

\begin{strictcode}
data Arg stream monad symbol t u =
   Arg { ord :: symbol -> Int,

         {- type arguments -}
         t :: monad t,
         u :: monad u,

         {- action arguments -}
         astar :: LexInfo stream symbol -> monad t,
         bstar :: LexInfo stream symbol -> monad u }

foo :: Streamable stream monad
       => Arg stream monad symbol t u -> stream symbol -> monad t
bar :: Streamable stream monad
       => Arg stream monad symbol t u -> stream symbol -> monad u
endstrictcode

Observe that the types of the actions {\tt astar} and {\tt bstar}, and
the lexing functions {\tt foo} and {\tt bar}, now end in \cd{monad
t} and \cd{monad u}, intead of {\tt t} and {\tt u}.

A worked example of how to use this appears in Section~\ref{sec:example-hs}.



\subsection{The automaton listing}

In the generated file, embedded as a comment immediately after the
functor's interface specification, is a listing of each lexing
function's state machine.  For each lexing function, the listing gives
the initial state and the total number of states, and for each state,
it gives the state that results from each possible input symbol.
Symbols that are not listed, implicitly transition to the error
state.



\section{CM-Yacc}

CM-Yacc is an LALR(1) parser generator for Standard ML and Haskell.
It converts a specification file into an SML or Haskell source-code
file that implements a parser functor.  The programmer may then
instantiate that functor with appropriate code for lexing actions to
obtain a parser.


\subsection{The specification file}

A CM-Yacc specification consists of a series of directives:

\begin{itemize}
\item
\cd{name \nonterm{identifier}}

Specifies the name of the generated functor for SML parsers,
and the name of the generated module for Haskell parsers.  This is
a required directive.

In Haskell specifications, the name must be a valid module
identifier, so it must begin with a capital letter.  In SML
specifications the name is not required to be capitalized, but it
is standard practice in SML code to capitalize functor names
nonetheless.

\item
\cd{terminal \nonterm{name} \lb of \nonterm{type name} \rb \lb \nonterm{precedence} \rb}

Declares a terminal.  If the ``\cd{of \nonterm{type name}}'' form is
used, then the terminal carries a value of type \nonterm{type name}.
Otherwise it carries no value.  If the \nonterm{precedence} form is
used, the terminal is marked with a precedence, which is used to
resolve shift-reduce conflicts.

\item
\cd{nonterminal \nonterm{name} : \nonterm{type name} =
\nonterm{production} \dots{} \nonterm{production}}

Declares a nonterminal carrying a value of type \nonterm{type name}.
The nonterminal expands using the productions in \nonterm{production
list}.

\item
\cd{start \nonterm{name}}

Indicates that the nonterminal \nonterm{name} is the start symbol.
This is a required directive.

\item
\cd{follower \nonterm{name}}

Indicates that the terminal \nonterm{name} is permitted to follow
the start symbol.  (In other words, it is permitted to be the first
symbol after a complete parse.)  If no followers are given, only the
end of the stream is permitted to follow the start symbol.

\item
\cd{data \nonterm{identifier}}

{\em Haskell specifications only:}
Specifies the name that the tool should give to the datatype of
terminals.  This is a required directive.  (In Standard ML, the
datatype of terminals is always named \cd{terminal}.)

\item
{\tt monadic}

{\em Haskell specifications only:}
Specifies that CM-Lex should generate a monadic parser.
\end{itemize}

Comments are written ``\cd{/* {\it{}comment text} */}'' and may appear
anywhere in the specification.  Comments may be nested.


\subsubsection{Productions}

Each production has the form:

\begin{tabbing}
\quad \= \kill
\nonterm{production} ::=
\\ \>
\cd{\nonterm{right-hand-side} \cd{=>} \nonterm{action} \lb \nonterm{precedence} \rb}
\end{tabbing}

\begin{tabbing}
\quad \= \kill
\nonterm{right-hand-side} ::=
\\ \>
\cd{\nonterm{constituent} \ldots \nonterm{constituent}}
\end{tabbing}

\begin{tabbing}
\quad \= \kill
\nonterm{constituent} ::=
\\ \>
\nonterm{terminal-or-nonterminal}
\\ \>
\cd{\nonterm{label} : \nonterm{terminal-or-nonterminal}}
\\ \>
\cd{( \nonterm{constituent} )}
\end{tabbing}

\begin{tabbing}
\quad \= \kill
\nonterm{label} ::=
\\ \>
\nonterm{positive integer}
\\ \>
\nonterm{identifier}
\end{tabbing}

Each production gives a right-hand-side (made up of terminals and
nonterminals) into which the nonterminal being defined can expand.
The right-hand-side may be empty.  When a production is used, the
action is called to convert the data carried by the right-hand-side's
constituents into the data carried by the nonterminal.

The labels on constituents determine which constituents' data are
passed to the action.  If a constituent omits the label, it passes no
data to the action.  If no constituents pass data to the action, the
domain type of the action function will be {\tt unit} ({\tt ()} in
Haskell).  Otherwise, the labels determine the action's domain type.

A label is either a positive integer or an identifier.  In a single
production, all the labels must be of the same sort ({\em i.e.,} all
numbers or all identifiers).  If the labels are identifiers, the
constituents' data are passed in a record, and the labels are used as
the names of the record's fields.  If the labels are numbers, all the
numbers from $1$ to $n$ must be used, and the action's domain is an
$n$-tuple.  The numbers indicate the order in which the constituents'
data appear in the tuple.  {\em Except:} If only the label $1$ is
used, the datum is passed bare, not in a $1$-tuple.

For example, with the directive:

\begin{code}
nonterminal Production : production =
   2:Constituents ARROW 1:Ident 3:Precedence => single_production
\end{code}

\noindent
the action \cd{single_production} has type \cd{ident * constitutents *
precedence -> production} (assuming that \cd{Ident}, \cd{Constituents}, and
\cd{Precedence} are declared to carry types \cd{ident},
\cd{constitutents}, and \cd{precedence}, respectively).



\subsubsection{Precedence}

If precedences are used, the tool consults them to resolve
shift-reduce conflicts.  They have the form:

\begin{tabbing}
\nonterm{precedence} ::=
\\
\quad \=
{\tt precl} \nonterm{number} \quad
\= left associative with precedence \nonterm{number}
\\ \>
{\tt precr} \nonterm{number} \>
right associative with precedence \nonterm{number}
\\ \>
{\tt noprec} \>
no precedence
\end{tabbing}

Precedences range from 0 to 100, with higher numbers indicating higher
precedence ({\em i.e.,} binding more tightly).  For a terminal, a {\tt
noprec} precedence is equivalent to omitting the precedence.  For a
production, if no precedence is given, the production's precedence is
inferred from the precedence of its last terminal.  A production has no precedence
if it is given precedence {\tt noprec}, if it contains no terminals,
or if its last terminal does not have a precedence.



\subsection{Invocation}

CM-Yacc is invoked from the command line with the specification file's
name as its argument.  The desired output file name can be specified
with the \cd{-o} flag.  Otherwise, the output file name is derived from from
the input file name by appending the appropriate suffix.

\begin{code}
cmyacc foo.cmyacc               # generates foo.cmyacc.sml
cmyacc foo.cmyacc -o bar.sml    # generates bar.sml

cmyacc-hs foo.cmyacc            # generates foo.cmyacc.hs
cmyacc-hs foo.cmyacc -o bar.hs  # generates bar.hs
\end{code}



\subsection{The generated functor (Standard ML)}
\label{sec:yacc-functor-sml}

CM-Yacc generates a functor from the given specification file.  For example,
consider the specification:

\begin{code}
name ParserFun

terminal NUMBER of t
terminal PLUS
terminal TIMES

nonterminal Factor : t =
  1:NUMBER => number_factor
  1:NUMBER TIMES 2:Factor => times_factor

nonterminal Term : t =
  1:Factor => factor_term
  1:Factor PLUS 2:Term => plus_term

start Term
\end{code}

CM-Yacc generates the functor:

\begin{strictcode}
functor ParserFun
   (structure Streamable : STREAMABLE
    structure Arg :
       sig
          type t

          val plus_term : t * t -> t
          val factor_term : t -> t
          val times_factor : t * t -> t
          val number_factor : t -> t

          datatype terminal =
             NUMBER of t
           | PLUS
           | TIMES

          val error : terminal Streamable.t -> exn
       end)
   :>
   sig
      val parse : Arg.terminal Streamable.t -> Arg.t * Arg.terminal Streamable.t
   end
= itdots implementation dots
endstrictcode

As with CM-Lex, the functor takes two arguments, a {\tt STREAMABLE}
and an argument that implements the parser's actions.  The {\tt Arg}
structure also must supply the datatype that implements terminals, and
it must supply an error function.  The error function is called when
the parser detects a syntax error.  It is passed the stream of
terminals beginning with the terminal that generated the error, and it
should return an {\tt exn}, which the parser then raises.

The result of the functor is a function {\tt parse}.  It takes in a
stream of terminals, and it returns a pair: the data carried by the
start symbol, and the stream of terminals that follows the complete
parse.  (If no followers are specified, the stream that follows the
parse is always empty.)


\subsection{The generated functor (Haskell)}

The generated code is an ersatz functor, similar to the one generated
by CM-Lex (Section~\ref{sec:lex-functor-hs}).  For example, consider
the specification:

\begin{code}
name ParserFun

terminal NUMBER of t
terminal PLUS
terminal TIMES

nonterminal Factor : t =
  1:NUMBER => number_factor
  1:NUMBER TIMES 2:Factor => times_factor

nonterminal Term : t =
  1:Factor => factor_term
  1:Factor PLUS 2:Term => plus_term

start Term

data Terminal
\end{code}

CM-Yacc generates the functor:\footnote{For readability, we have
used unqualified names, but the tool actually uses fully qualified names.}

\begin{strictcode}
data Terminal t =
   NUMBER t
 | PLUS
 | TIMES

data Arg stream t =
   Arg { error :: stream (Terminal t) -> SomeException,

         {- type arguments -}
         t :: t

         {- action arguments -}
         plus_term :: t -> t -> t,
         factor_term :: t -> t,
         times_factor :: t -> t -> t,
         number_factor :: t -> t }

parse :: Streamable stream Identity 
         => Arg stream t -> stream (Terminal t) -> (t, stream (Terminal t))
endstrictcode

If the \cd{monadic} directive is added to the specification, CM-Yacc
instead generates:

\begin{strictcode}
data Terminal t =
   NUMBER t
 | PLUS
 | TIMES

data Arg stream monad t =
   Arg { error :: stream (Terminal t) -> monad SomeException,

         {- type arguments -}
         t :: monad t

         {- action arguments -}
         plus_term :: t -> t -> t,
         factor_term :: t -> t,
         times_factor :: t -> t -> t,
         number_factor :: t -> t }

parse :: Streamable stream monad 
         => Arg stream monad t 
               -> stream (Terminal t) -> monad (t, stream (Terminal t))
endstrictcode

A worked example of how to use this appears in
Section~\ref{sec:example-hs}.



\subsection{The automaton listing}

In the generated file, embedded as a comment immediately after the
functor's interface specification, is a listing of the parsing
function's state machine.  Its form should be familiar to users of
Yacc's automaton listings.

The bulk of the listing is a list of the automaton's states.  For
example:

\begin{code}
State 4:

0 : Factor -> . NUMBER  / 1
1 : Factor -> . NUMBER TIMES Factor  / 1
2 : Term -> . Factor  / 0
3 : Term -> . Factor PLUS Term  / 0
3 : Term -> Factor PLUS . Term  / 0

NUMBER => shift 2
Factor => goto 1
Term => goto 6
\end{code}

The state begins with a list of LR(1) items, each of which gives one
scenario in which the parser could be in the state.  For example, the
item

\begin{code}
3 : Term -> Factor PLUS . Term  / 0
\end{code}

\noindent
means that the parser could be in the process of recognizing
\cd{Factor PLUS Term} which it would then reduce to \cd{Term} using
the 3rd production (counting from zero).  The dot is a cursor,
indicating that so far it has seen
\cd{Factor PLUS}.  The ``\cd{/ 0}'' gives the production's lookahead
set: the set of terminals that can follow this production.

The lookahead sets are given after the state listings, such as:

\begin{code}
lookahead 0 = $ 
lookahead 1 = $ PLUS 
\end{code}

The \cd{$} is a special terminal representing the end of the stream.
In lookahead set 0, only the end of the stream can follow the
production.  In lookahead set 1, the \cd{PLUS} terminal can also
follow it.

The LR(1) items are given to help the user understand the parser the
tool has constructed.  Following the LR(1) is the state's action table.
For each terminal, the action table indicates shift, reduce, or error.
(Error transitions are omitted from the listing.)  In a \cd{shift}~$n$
transition, the parser consumes the terminal and transitions to state
$n$.  In a \cd{reduce}~$n$ transition, the parser leaves the terminal on the
stream, reduces using production $n$, and retrieves an old state from
its stack.

For example, in the state

\begin{code}
State 2:

0 : Factor -> NUMBER .  / 1
1 : Factor -> NUMBER . TIMES Factor  / 1

$ => reduce 0
PLUS => reduce 0
TIMES => shift 5
\end{code}

\noindent
the parser will reduce using production 0 if it sees either
end-of-stream or \cd{PLUS}.  If it sees \cd{TIMES} it will shift it
and transition to a state that includes the item:

\begin{code}
1 : Factor -> NUMBER TIMES . Factor  / 1
\end{code}

Following the action table is the state's goto table.  When a state is
re-entered after a reduce, the goto table tells which state the parser
should enter, depending on which nonterminal was just reduced.


\subsection{Shift-reduce conflicts}

When a grammar is ambiguous, an action table may list multiple
possible actions.  For example, consider:

\begin{code}
name ParserFun

terminal NUMBER of t
terminal PLUS
terminal TIMES precl 0

nonterminal Term : t =
  1:NUMBER => number_term
  1:Term PLUS 2:Term => plus_term
  1:Term TIMES 2:Term => times_factor

start Term
\end{code}

The grammar is ambiguous, and it declares that \cd{TIMES} is left
associative, but it gives no precedence for \cd{PLUS}.  Its automaton
contains the state:

\begin{code}
State 5:

1 : Term -> Term . PLUS Term  / 1
2 : Term -> Term . TIMES Term  / 1
2 : Term -> Term TIMES Term .  / 1

$ => reduce 2
PLUS => shift 4, reduce 2  CONFLICT
TIMES => reduce 2, shift 3  PRECEDENCE
\end{code}

In this state, only the end-of-stream action is unambiguous.  When
\cd{PLUS} is seen, the parser could either shift, or reduce using
production 2.  The actual generated functor uses the first action
listed.  When \cd{TIMES} is seen, the parser could either shift or
reduce, but the left associativity of TIMES indicates that the parser
should resolve the conflict in favor of the reduce.

Thus, the {\tt CONFLICT} notation marks an unresolved ambiguity, while
the {\tt PRECEDENCE} notation marks an ambiguity resolved by
precedence.

Whenever there is an unresolved shift-reduce conflict, the tool
prefers the shift.  When there is a reduce-reduce conflict, the tool
chooses an arbitrary production.  Reduce-reduce conflicts usually
indicate a serious problem with the grammar, so once one has been
detected, the tool no longer uses precedence to resolve shift-reduce
conflicts.


\section{A Standard ML example}
\label{sec:example-sml}

As a more realistic example, we implement a calculator that processes
an input stream and returns its value.  For simplicity, the calculator
stops at the first illegal character (which might be the end of the
stream).  The lexer specification is:

\begin{code}
name CalcLexFun
alphabet 128

set digit = (range '0 '9)
set whitechar = (or 32 9 10)  /* space, tab, lf */

function lex : t =
  (+ digit) => number
  '+ => plus
  '* => times
  '( => lparen
  ') => rparen
  (+ whitechar) => whitespace

  /* Stop at the first illegal character */
  epsilon => eof
\end{code}

\noindent
which generates the functor:

\begin{strictcode}
functor CalcLexFun
   (structure Streamable : STREAMABLE
    structure Arg :
       sig
          type symbol
          val ord : symbol -> int

          type t

          type self = { lex : symbol Streamable.t -> t }
          type info = { match : symbol list,
                        len : int,
                        start : symbol Streamable.t,
                        follow : symbol Streamable.t,
                        self : self }

          val eof : info -> t
          val lparen : info -> t
          val number : info -> t
          val plus : info -> t
          val rparen : info -> t
          val times : info -> t
          val whitespace : info -> t
       end)
   :>
   sig
      val lex : Arg.symbol Streamable.t -> Arg.t
   end
= itdots implementation dots
endstrictcode

The parser specification is:

\begin{code}
name CalcParseFun

terminal NUMBER of t
terminal PLUS
terminal TIMES
terminal LPAREN
terminal RPAREN

nonterminal Atom : t =
  1:NUMBER => number_atom
  LPAREN 1:Term RPAREN => paren_atom

nonterminal Factor : t =
  1:Atom => atom_factor
  1:Atom TIMES 2:Factor => times_factor

nonterminal Term : t =
  1:Factor => factor_term
  1:Factor PLUS 2:Term => plus_term

start Term
\end{code}

\noindent
which generates the functor:

\begin{strictcode}
functor CalcParseFun
   (structure Streamable : STREAMABLE
    structure Arg :
       sig
          type t

          val plus_term : t * t -> t
          val factor_term : t -> t
          val times_factor : t * t -> t
          val atom_factor : t -> t
          val paren_atom : t -> t
          val number_atom : t -> t

          datatype terminal =
             NUMBER of t
           | PLUS
           | TIMES
           | LPAREN
           | RPAREN

          val error : terminal Streamable.t -> exn
       end)
   :>
   sig
      val parse : Arg.terminal Streamable.t -> Arg.t * Arg.terminal Streamable.t
   end
= itdots implementation dots
endstrictcode

We then assemble the calculator as follows:

\begin{bigstrictcode}

structure Calculator
  :>
  sig
    val calc : char Stream.stream -> int
  end
  =
  struct
     open Stream

     datatype terminal =
        NUMBER of int
      | PLUS
      | TIMES
      | LPAREN
      | RPAREN

     structure Lexer =
        CalcLexFun
        (structure Streamable = StreamStreamable
         structure Arg =
            struct
               type symbol = char
               val ord = Char.ord

               type t = terminal front
    
               type self = { lex : char stream -> t }
               type info = { match : char list,
                             follow : char stream,
                             self : self,
                             len : int,
                             start : char stream }
    
               fun number ({ match, follow, self, ... }:info) =
                 Cons (NUMBER (Option.valOf (Int.fromString (String.implode match))),
                       lazy (fn () => #lex self follow))
    
               fun simple terminal ({ follow, self, ... }:info) =
                 Cons (terminal, lazy (fn () => #lex self follow))
    
               val plus = simple PLUS
               val times = simple TIMES
               val lparen = simple LPAREN
               val rparen = simple RPAREN
    
               fun whitespace ({ follow, self, ...}:info) =
                 #lex self follow
    
               fun eof _ = Nil
             end)
     
      structure Parser =
         CalcParseFun
         (structure Streamable = StreamStreamable
          structure Arg =
             struct
               type t = int
    
               fun id x = x
    
               val number_atom = id
               val paren_atom = id
               val atom_factor = id
               fun times_factor (x, y) = x * y
               val factor_term = id
               fun plus_term (x, y) = x + y
    
               datatype terminal = datatype terminal
    
               fun error _ = Fail "syntax error"
             end)
    
      fun calc strm = #1 (Parser.parse (lazy (fn () => Lexer.lex strm)))
  end
endbigstrictcode



\section{A Haskell example}
\label{sec:example-hs}

Here we show how to build the calculator example in Haskell.  The
lexer specification is:

\begin{code}
name CalcLexFun
alphabet 128

set digit = (range '0 '9)
set whitechar = (or 32 9 10)  /* space, tab, lf */

function lex : t =
  (+ digit) => number
  '+ => plus
  '* => times
  '( => lparen
  ') => rparen
  (+ whitechar) => whitespace

  /* Stop at the first illegal character */
  epsilon => eof

monadic
\end{code}

\noindent
which generates a module exporting:

\begin{strictcode}
data Arg stream monad symbol t =
   Arg { ord :: symbol -> Int,

         {- type arguments -}
         t :: monad t,

         {- action arguments -}
         eof :: LexEngine.LexInfo stream symbol -> monad t,
         lparen :: LexEngine.LexInfo stream symbol -> monad t,
         number :: LexEngine.LexInfo stream symbol -> monad t,
         plus :: LexEngine.LexInfo stream symbol -> monad t,
         rparen :: LexEngine.LexInfo stream symbol -> monad t,
         times :: LexEngine.LexInfo stream symbol -> monad t,
         whitespace :: LexEngine.LexInfo stream symbol -> monad t }

lex :: LexEngine.Streamable stream monad
       => CalcLexFun.Arg stream monad symbol t -> stream symbol -> monad t
endstrictcode

The parser specification is:

\begin{code}
name CalcParseFun

terminal NUMBER of t
terminal PLUS
terminal TIMES
terminal LPAREN
terminal RPAREN

nonterminal Atom : t =
  1:NUMBER => number_atom
  LPAREN 1:Term RPAREN => paren_atom

nonterminal Factor : t =
  1:Atom => atom_factor
  1:Atom TIMES 2:Factor => times_factor

nonterminal Term : t =
  1:Factor => factor_term
  1:Factor PLUS 2:Term => plus_term

start Term

data Terminal

monadic
\end{code}

\noindent
which generates a module exporting:

\begin{strictcode}
data Terminal t =
   NUMBER t
 | PLUS
 | TIMES
 | LPAREN
 | RPAREN

data Arg stream monad t =
   Arg { error :: stream (Terminal t) -> monad Control.Exception.SomeException,

         {- type arguments -}
         t :: monad t

         {- action arguments -}
         plus_term :: t -> t -> t,
         factor_term :: t -> t,
         times_factor :: t -> t -> t,
         atom_factor :: t -> t,
         paren_atom :: t -> t,
         number_atom :: t -> t }

parse :: ParseEngine.Streamable stream monad 
         => CalcParseFun.Arg stream monad t -> stream (Terminal t)
               -> monad (t, stream (Terminal t))
endstrictcode

We then assemble the calculator as follows:

\begin{bigstrictcode}
module Calc where

import Data.Char as Char
import Control.Exception
import Util.Stream
import Util.LexEngine
import CalcLexFun as Lex
import CalcParseFun as Parse

type Term = Terminal Int


{- The lexer -}

simple :: Term -> LexInfo (Stream IO) Char -> IO (Front Term (Stream IO Term))
simple terminal info =
   do {
      t <- lazy (Calc.lex (follow info));
      return (Cons terminal t)
      }


lexarg =
   Lex.Arg
      {
      Lex.ord = Char.ord,

      Lex.t = undefined :: IO (Front Term (Stream IO Term)),

      Lex.number =
         (\ info ->
             do {
                t <- lazy (Calc.lex (follow info));
                return (Cons (NUMBER (read (match info))) t)
                }),

      Lex.lparen = simple LPAREN,
      Lex.rparen = simple RPAREN,
      Lex.plus = simple PLUS,
      Lex.times = simple TIMES,

      Lex.whitespace =
         (\ info -> Calc.lex (follow info)),

      Lex.eof =
         (\ info -> return Nil )
      }


lex :: Stream IO Char -> IO (Front Term (Stream IO Term))
lex s = Lex.lex lexarg s


{- The parser -}

newtype SyntaxError = SyntaxError (Stream IO Term)
instance Show SyntaxError where
   show _ = "syntax error"
instance Exception SyntaxError


parsearg =
   Parse.Arg
      {
      Parse.error =
         (\ s -> return (toException (SyntaxError s))),

      Parse.t = undefined :: IO Int,

      Parse.plus_term = (+),
      Parse.times_factor = (*),

      Parse.number_atom = id,
      Parse.paren_atom = id,
      Parse.atom_factor = id,
      Parse.factor_term = id
      }


calc :: Stream IO Char -> IO Int
calc s =
   do {
      s <- lazy (Calc.lex s);
      (x, _) <- Parse.parse parsearg s;
      return x
      }
endbigstrictcode


\appendix
\section{Installation}

To install CM-Lex and CM-Yacc, obtain the source code either from the
distribution at:

\begin{center}
\url{www.cs.cmu.edu/~crary/cmtool/}
\end{center}

\noindent
or from Github at \url{kcrary/cmtool}.  (The project uses a submodule,
so if you clone the project from Github, use the \cd{--recursive}
option.)

Then follow the instructions in the INSTALL file.  On Unix platforms
you will need either MLton or Standard ML of New Jersey installed.  On
Windows platforms you will need Standard ML of New Jersey.  This
manual is included as \url{manual.pdf}.

\end{document}
